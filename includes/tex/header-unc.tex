
%Font packages
%\usepackage[T1]{fontenc}
%\usepackage[utf8]{inputenc}
\usepackage{csquotes}
\usepackage[english]{babel}
\usepackage[utf8]{inputenc}

\DeclareUnicodeCharacter{2212}{-}

% List of acronyms
\usepackage{longtable}
%\usepackage[acronym]{glossaries}%NOTE: this will NOT work in markdown to latex. have to have access to the latex file with same name to access the glossary.

\usepackage{lscape}
\usepackage{tipa}

%% Extra packages that I added -----------------
\usepackage{textgreek}
\usepackage[table]{xcolor}
\usepackage{fancyhdr}
\usepackage{booktabs}
\usepackage{setspace}
\usepackage{kvoptions}
%\usepackage{afterpage}
\usepackage{rotating}%for sidewaystable in xtable
\usepackage{hyperref}% to highlight links in different colors
\hypersetup{
    colorlinks=true,
    linkcolor=blue,
    filecolor=magenta,      
    urlcolor=cyan,
}

% see the following link for info on biblatex sort order issue: 
% http://tex.stackexchange.com/questions/51434/biblatex-citation-order
\usepackage[style=numeric,
        hyperref=true,
        maxbibnames=99,
        firstinits=true,
        uniquename=init,
        doi=true,
        backref=true]{biblatex}

\usepackage{float}% for placement of figures and tables

\usepackage{tikz}%for DAG
\usetikzlibrary{arrows.meta,positioning}

\usepackage{array}%for smaller landscape tables
\usepackage{graphicx}

\newcolumntype{Z}[1]{>{\raggedright\let\newline\\\arraybackslash\hspace{0pt}}m{#1}}
% \newcolumntype{C}[1]{>{\centering\let\newline\\\arraybackslash\hspace{0pt}}m{#1}}
\newcolumntype{R}[1]{>{\raggedleft\let\newline\\\arraybackslash\hspace{0pt}}m{#1}}

\usepackage{tabulary}

\usepackage[bf,singlelinecheck=off]{caption}
\usepackage{eso-pic,graphicx,transparent}% see http://stackoverflow.com/questions/32748248/watermark-in-rmarkdown
%%%%%%%%%%%%%%%%%%%%%%%%%%%%%%%%%%%%%%%%%%%%%%%%%%%%%%%%%%%%%
% GLOSSARIES AND ABBREVIATIONS
%%%%%%%%%%%%%%%%%%%%%%%%%%%%%%%%%%%%%%%%%%%%%%%%%%%%%%%%%%%%%
% To update the printed glossary, you need to run:
% - pdflatex dissertation
% - makeglossaries dissertation
% - pdflatex dissertation
% On Windows, you might need to install Perl first.

%% MY own note: Since I am doing this in Rmd and not LaTex need to make a separate
%% LaTex file, glossary.tex, and add the acronyms there.

%\newacronym{unc}{UNC}{The University of North Carolina at Chapel Hill}
%\makeglossaries

% NOTE: the glossaries package in LaTex does not work when using markdown. Trying another approach from "Using LaTex to Write a PhD Thesis" by Nicola L.C. Talbot



\usepackage{datagidx}% NOTE: this will not work unless you disable glossaries package


\newgidx{glossary}{Glossary}
\newgidx{acronym}{List of Abbreviations}

\DTLgidxSetDefaultDB{acronym}
\newacro{TC}{Total Cholesterol}
\newacro{HDL-C}{High Density Lipoprotein Cholesterol}
\newacro{LDL-C}{Low Density Lipoprotein Cholesterol}
\newacro{TG}{triglycerides}
\newacro{GWAS}{Genome-Wide Association Studies}
\newacro{GRS}{Genetic Risk Score}
\newacro{PRS}{Polygenic Risk Score}
\newacro{non-HDL-C}{non-High Density Lipoprotein Cholesterol}
\newacro{BMI}{body mass index}
\newacro{POF}{Postnatal Overfeeding}
\newacro{RNA}{Ribonucleic Acid}
\newacro{ALT}{Alanine aminotransferase}
\newacro{GGT}{Gamma glutamyltransferase}
\newacro{NAFLD}{nonalcoholic fatty liver disease}
\newacro{LBW}{low birthweight}
\newacro{SITAR}{SuperImposition by Translation and Rotation}
\newacro{LGMM}{latent growth mixture model}
\newacro{SLS}{Santiago Longitudinal Study}
\newacro{WFL}{weight-for-length}
\newacro{HL}{Hispanic/Latino}
\newacro{CVD}{Cardiovascular Disease}
\newacro{ASCVD}{Atherosclerotic Cardiovascular Disease}
\newacro{DOHaD}{Developmental Origins of Health and Disease}
\newacro{cGXE}{candidate gene x environment}
\newacro{GXE}{genotype x environment}
\newacro{SNP}{single nucleotide polymorphism}
\newacro{DAG}{directed acyclic diagram}
\newacro{LD}{linkage disequilibrium}
\newacro{AA}{African American}
\newacro{GLGC}{Global Lipid Genetic Consortium}
\newacro{EA}{European ancestry}
\newacro{IDL}{intermediate-density lipoprotein}
\newacro{VLDL}{very low-density lipoprotein}
\newacro{MA}{Mexican American}
\newacro{NHW}{non-Hispanic White}
\newacro{GOCS}{Growth and Obesity Cohort Study}
\newacro{CV}{coefficient of variation}
\newacro{IS}{iron sufficiency}
\newacro{LCFA}{longitudinal confirmatory factor analysis}
\newacro{CFA}{confirmatory factor analysis}
\newacro{CAD}{coronary artery disease}
\newacro{CAC}{coronary artery calcification}
\newacro{MAF}{minor allele frequency}

% override default description on this one:
\newterm
[%
  symbol={\ensuremath{LXR\alpha}},%symbol for name
  long={liver X receptor alpha}, % a brief description
  short={\ensuremath{LXR\alpha}},
  description={liver X receptor alpha}, % a brief description
  text={\ensuremath{LXR\alpha}}
]{LXR-alpha} %name

\newterm
[%
  symbol={PPAR\ensuremath{\gamma}},
  long={peroxisome proliferator-activated receptor gamma}, % a brief description
  short={PPAR\ensuremath{\gamma}},
  description={peroxisome proliferator-activated receptor gamma}, % a brief description
  text={PPAR\ensuremath{\gamma}},
  sort={PPAR\ensuremath{\gamma}}
]{PPAR-gamma}

%\newacro[text=LXR\ensuremath{\alpha}]{lxralpha}{liver X receptor alpha}
%\newacro[LXR\ensuremath{$\alpha$}]{}{lxralpha}{lxralpha}
%\newacro{PPAR\ensuremath{\gamma}}{peroxisome proliferator-activated receptor gamma}
\newacro{mRNA}{messenger ribonucleic acid}
\newacro{DNA}{deoxyribonucleic acid}
\newacro{PNOF}{postnatal overfeeding}
%no need to call makeglossaries.


% \usepackage{datagidx}% NOTE: this will not work unless you disable glossaries package
%  \newgidx{glossary}{Glossary}
%  \newgidx{acronym}{List of Abbreviations}
%  
%  \DTLgidxSetDefaultDB{acronym}
%  \newacro{TC}{Total Cholesterol}
%no need to call makeglossaries.


% \newgidx{glossary}{Glossary}
% \newgidx{acronym}{List of Abbreviations}


% \DTLgidxSetDefaultDB{glossary}
% \newterm
% [%
% description={Total cholesterol},% brief description
% ]%
% {tc}% the name



%% Math Packages %%%%%%%%%%%%%%%%%%%%%%%%%%%%%%%%%%%%%%%%%%%%
\usepackage{amsmath}
\usepackage{amsthm}
\usepackage{amsfonts}
\usepackage{bbm}
\usepackage{amssymb}
\usepackage{geometry}

%% NOTE: I commented this section out. Not sure why but it kept my markdown file from compiling, kicking up a 'error 43'. Do I need to put this back? not sure.
%% Reduce spacing between paragraph and section title %%%%%%%
%% @todo: Put this modification in the class file itself.
% \usepackage{titlesec}
% \titlespacing*{\section}
% {0pt}{-5pt}{0pt}
% \titlespacing*{\subsection}
% {0pt}{-5pt}{0pt}
\usepackage{indentfirst}   %Indents first paragraphs in every section.

\setlength\parindent{24pt}

%% Flush footnotes to the left
\usepackage[hang,flushmargin]{footmisc}
%% Places footnotes immediately below horizontal rule
\setlength{\footnotesep}{0pt}

%% Normal LaTeX or pdfLaTeX? %%%%%%%%%%%%%%%%%%%%%%%%%%%%%%%%
\RequirePackage{ifpdf}

% %% Packages for Graphics & Figures %%%%%%%%%%%%%%%%%%%%%%%%%%
% \ifpdf %%Inclusion of graphics via \includegraphics{file}
% 	\usepackage[pdftex]{graphicx} %%graphics in pdfLaTeX
% \else
% 	\usepackage[dvips]{graphicx} %%graphics and normal LaTeX
% \fi


\usepackage{adjustbox}